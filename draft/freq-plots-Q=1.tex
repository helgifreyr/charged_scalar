\begin{figure*}
  \centering
  \begin{subfigure}[b]{0.347\textwidth}
    \centering
    \includegraphics[width=\textwidth]{{I-mu-0.1-Q-all-q-0.9-Q}.eps}
    \caption{$q=0.9$}
    \label{fig:I-Q1.0-q0.9}
  \end{subfigure}%
  ~
  \begin{subfigure}[b]{0.33\textwidth}
    \centering
    \includegraphics[width=\textwidth]{{I-mu-0.1-Q-all-q-1.5-Q}.eps}
    \caption{$q=1.5$}
    \label{fig:I-Q1.0-q1.5}
  \end{subfigure}%
  ~
  \begin{subfigure}[b]{0.33\textwidth}
    \centering
    \includegraphics[width=\textwidth]{{I-mu-0.1-Q-all-q-2.0-Q}.eps}
    \caption{$q=2.0$}
    \label{fig:I-Q1.0-q2.0}
  \end{subfigure}%
  \caption{The imaginary and real part of $\omega$ drawn as a function of the mirror radius $r_m$ for various values of the black hole charge, $Q$, and the scalar charge, $q$. The scalar mass is fixed at $\mu=0.1$.}\label{fig:freq-plots-Q}
\end{figure*}
