\begin{figure*}
  \centering
  \begin{subfigure}[b]{0.5\textwidth}
    \centering
    \includegraphics[width=\textwidth]{{I-Q-all-ratio-1.0}.eps}
    % \caption{$\frac{q}{\mu}=1.0$}
    \label{fig:ratio1.0}
  \end{subfigure}%
  ~
  \begin{subfigure}[b]{0.5\textwidth}
    \centering
    \includegraphics[width=\textwidth]{{I-Q-all-ratio-1.2}.eps}
    % \caption{$\frac{q}{\mu}=1.2$}
    \label{fig:ratio1.2}
  \end{subfigure}
  \\
  \begin{subfigure}[b]{0.5\textwidth}
    \centering
    \includegraphics[width=\textwidth]{{I-Q-all-ratio-1.4}.eps}
    % \caption{$\frac{q}{\mu}=1.4$}
    \label{fig:ratio1.4}
  \end{subfigure}%
  ~
  \begin{subfigure}[b]{0.5\textwidth}
    \centering
    \includegraphics[width=\textwidth]{{I-Q-all-ratio-2.0}.eps}
    % \caption{$\frac{q}{\mu}=2.0$}
    \label{fig:ratio2.0}
  \end{subfigure}
  \caption{The imaginary part of the frequency plotted versus the radius of the mirror for various ratios of the scalar charge, $q$. The scalar mass has been fixed at$\mu=0.3$.}\label{fig:freq-plots-nolog}
\end{figure*}
