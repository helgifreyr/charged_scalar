%
% filename = chargedscalar.tex
%
% First version: March 2013
% Last revision: 
%
%-----------------------------------------------------------------------

\documentclass[aps, prd, twocolumn, amsmath, floats,floatfix, superscriptaddress,
nofootinbib, showpacs]{revtex4-1}


%%%%%%%%%%%%%%%%%%%%%%%%%
%%%   LOAD PACKAGES   %%%
%%%%%%%%%%%%%%%%%%%%%%%%%
 
 
\usepackage{amssymb}
\usepackage{amsmath}
\usepackage{verbatim}
\usepackage{mathrsfs}
\usepackage{amsfonts}
\usepackage{latexsym}
\usepackage{epsfig}
\usepackage{color}
\usepackage{graphicx}
\usepackage{units}
\usepackage{slashbox}
\usepackage{caption,subcaption}


\newcommand{\Real}{\mathbb{R}}
\newcommand{\Complex}{\mathbb{C}}
\newcommand{\re}{\mbox{Re}}
\newcommand{\im}{\mbox{Im}}
\def\del{\partial}
\def\be{\begin{equation}}
\def\ee{\end{equation}}
\def\beq{\begin{eqnarray}}
\def\eeq{\end{eqnarray}}
\def\non{\nonumber}
\def\L{\mathcal{L}}
\def\p{\partial}


%%%%%%%%%%%%%%%%%%%%%%%%%
%%%   BEGIN DOCUMENT  %%%
%%%%%%%%%%%%%%%%%%%%%%%%%

\begin{document}


\renewcommand{\t}{\times}


%%%%%%%%%%%%%%%%%

%%%   TITLE   %%%
%%%%%%%%%%%%%%%%%
%\title{Growing scalar field modes in charged black holes}  
\title{Superradiant instabilities in charged black holes}  

 \author{Juan~Carlos~Degollado}\email{jcdaza@ua.pt}
   \affiliation{
   Departamento de F\'\i sica da Universidade de Aveiro and I3N, 
   Campus de Santiago, 3810-183 Aveiro, Portugal.
 }
 
\author{Carlos~A.~R.~Herdeiro}\email{herdeiro@ua.pt}
   \affiliation{
   Departamento de F\'\i sica da Universidade de Aveiro and I3N, 
   Campus de Santiago, 3810-183 Aveiro, Portugal.
}

\author{Helgi Freyr R\'unarsson} \email{helgi.runarsson@ua.pt}
   \affiliation{
   Departamento de F\'\i sica da Universidade de Aveiro and I3N, 
   Campus de Santiago, 3810-183 Aveiro, Portugal.
}


%%%%%%%%%%%%%%%%
%%%   DATE   %%%
%%%%%%%%%%%%%%%%

\date{\today}


%%%%%%%%%%%%%%%%%%%%
%%%   ABSTRACT   %%%
%%%%%%%%%%%%%%%%%%%%

\begin{abstract}  
We study the minimally coupled Klein-Gordon equation for a charged scalar field in the 
background of a Reissner-Nordstr\"om black hole.
\end{abstract}

%with a characteristic decaying time


%%%%%%%%%%%%%%%%
%%%   PACS   %%%
%%%%%%%%%%%%%%%%

\pacs{
95.30.Sf  % relativity and gravitation
}


%%%%%%%%%%%%%%%%%%%%%%
%%%   MAKE TITLE   %%%
%%%%%%%%%%%%%%%%%%%%%%

\maketitle


%%%%%%%%%%%%%%%%%%%%%%%%%%%%%%%%%%%%%%%%%%%%%%%%%%
\section{Introduction.}
\label{sec:introduction}
%%%%%%%%%%%%%%%%%%%%%%%%%%%%%%%%%%%%%%%%%%%%%%%%%%
%Although is difficult to find physical situations where the electric charge of a black
%hole is significant, investigation of the Reissner-Nordstr\"om (RN) black hole does not
%lack of interest since it provides a more general framework that the Schwarzschild
%solution. The study of RN solution have contributed to our understanding of more
%complicated geometries.
%Indeed highly charged case black holes give us information about the rapidly rotating
%black hole. 
%The RN solution similar to Kerr has a inner horizon as well a event horizon.
%In this respect the effect of the charge is similar to that the angular momentum in the
%Kerr case. 

In classical relativistic gravity, black holes are observer independent space-time 
regions unable to communicate with their exterior \cite{Hawking:1973uf}. Thus, within this
description, information captured by black holes is trapped therein forever and cannot be
recovered by exterior observers. 
 
Given this picture it is intriguing, at first, to realise that there is a 
\textit{classical} process through which energy can be extracted from a black hole:
\textit{superradiant scattering}.  In one form, this process amounts to the amplification
of waves impinging on a Kerr black hole, provided the frequency $\omega$ and azimuthal
quantum number $m$ of the wave modes obey the condition $\omega<m\Omega_+$, where
$\Omega_+$ is the angular velocity of the outer Kerr horizon
\cite{Bardeen:1972fi,Starobinsky:1973a,Press:1972zz}. The extraction of energy and
consequent decrease of the black hole mass $M$ is, however, necessarily accompanied by the
extraction of angular momentum and consequent decrease of the  black hole spin $J$. In
fact, it was shown by Christodoulou \cite{Christodoulou:1970wf} that the particle analogue
of this  process - the Penrose process \cite{Penrose:1969pc} - is irreversible,
subsequently realised to mean that the black hole area never decreases
\cite{Christodoulou:1972kt}. Finally, the identification between black hole area and
entropy \cite{Bekenstein:1973ur,Bardeen:1973gs} made clear that it is only (rotational)
energy that is being extracted from the black hole, not information.  
  
In another form, superradiant scattering amounts to the amplification of \textit{charged} 
waves impinging on a Reissner-Nordstr\"om (RN) black hole, provided the frequency $\omega$
and the charge $q$ of the wave modes obey the condition $\omega<q\Phi_+$, where $\Phi_+$
is the electric potential of the outer Reissner-Nordstr\"om horizon
\cite{Bekenstein:1973mi}. The extraction of (Coulomb) energy and consequent decrease of
the black hole mass $M$ is, in this case, necessarily accompanied by the extraction of
charge and consequent decrease of the  black hole charge $Q$, such that, again, the
area/entropy of the RN black hole does not decrease.

The existence of superradiant modes can be converted into an \textit{instability} of the 
background if a mechanism to trap these modes in a vicinity of the black hole is provided:
heuristically, these modes are then recurrently scattered off the black hole and
amplified, eventually producing a non-negligible background back-reaction. This
possibility, antecipated by Zel'dovich \cite{Zeldovich:1971}, was named \textit{black hole
bomb} by Press and Teukolsky \cite{Press:1972zz} and has been studied extensively in the
Kerr case within the linear analysis
\cite{Cardoso:2004nk,Hod:2009cp,Rosa:2009ei,Pani:2012vp}. 

The unstable states found in the Kerr case are  localised in a potential well found
outside the potential barrier of the effective potential. The growth of such states can be
seen at linear level, but  a fully
non-linear study is required to address the end-point of this instability. This, however, 
has not yet been achieved, and will most certainly involve a numerical analysis 
\cite{Witek:2010qc}.  


Considerable less attention has been devoted to the charged case, perhaps due to the 
lack of astrophysical motivation. Moreover, the studies found in the literature
\cite{Furuhashi:2004jk,Hod:2012zz} discard the possibility of an instability by performing
an analysis of the effective potential and showing that no potential well is found for
quasi-bound states compatible with the superradiance condition. 



%%%%%%%%%%%%%%%%%%%%%%%%%%%%%%%%%%%%%%%%%%%%%%%%%%
\section{Quasi-bound states.}
\label{sec:bound-states}
%%%%%%%%%%%%%%%%%%%%%%%%%%%%%%%%%%%%%%%%%%%%%%%%%%
We shall consider a massive, charged scalar field, $\Phi$, with mass $\mu$ and charge $q$,
in the linear regime is described by the wave equation
\begin{equation}
\left[(D^\nu-iqA^\nu)(D_\nu-iqA_\nu)-\mu^2\right]\Phi=0 \ ,
\label{eq:we}
\end{equation}
propagating in the background of a Reissner-Nordstr\"om black hole with charge $Q$ and
mass $M$. Written in terms of Boyer-Linquist type coordinates the line element is 
\begin{equation}
ds^2=-f(r)dt^2+\frac{dr^2}{f(r)}+r^2(d\theta^2+\sin^2\theta d\phi^2) \ ,  
\end{equation}
where 
\be
f(r)=\frac{(r-r_+)(r-r_-)}{r^2} \ , \ \ r_\pm\equiv M\pm\sqrt{M^2-Q^2} \ ,
\ee
and $A=-Q/rdt$.

%%%%%%%%%%%%%%%%%%%%%%%%%%%%%%%%%%%%%%%%%%%%%%%%%%
\subsection{Properties of the Potential}
\label{sec:potential}
%%%%%%%%%%%%%%%%%%%%%%%%%%%%%%%%%%%%%%%%%%%%%%%%%%
In Fig.~\ref{fig:pot_char} we plot the effective potential for several values of the
scalar charge while keeping the other parameters fixed. The asymptotic value of the
potential at infinity is $\mu^2$ and for $r\rightarrow r_{+} (r* \rightarrow -\infty)$
$V_{eff}$ tends to a constant $\omega_c(2\omega-\omega_c) = qQ/r_{+} (2\omega -
qQ/r_{+})$. This depends on the value of the frequency, which is unknown, however we used
an arbitrary value to make clear the behaviour of the potential close the horizon. 

\begin{figure}[!ht]
%\includegraphics[width=0.47\textwidth]{m0.4_Q0.9_qs_w0.01.eps}
\includegraphics[width=0.47\textwidth]{m0.4_Q0.8_qs_w0.39.eps} 
\caption{Effective potential for some scalar charges. The values of the other
parameters are: $M=1,\,Q=0.9,\, \mu=0.4,\, \omega=0.39, \ell=1$. The  asymptotic value 
of the potential close the outer horizon is given by $\omega_c(2\omega-\omega_c)$. The
height of the centrifugal barrier increases with the charge of the field; the constant 
value of the potential near the outer horizon also increases with the charge of the field
but only up to some maximum; then it starts decreasing, in accordance with the quadratic
behaviour in $q$ present in $\omega_c(2\omega-\omega_c)$.}
\label{fig:pot_char}
\end{figure}

Figure~\ref{fig:pot_hbchar} \  shows the dependence of the potential on the background
charge $Q$. The value of the peak increase with the charge. Only for small enough $Q$ a
potential well is seen. 



\begin{figure}[!ht]
\includegraphics[width=0.47\textwidth]{m0.4_Qs_q0.5_w0.39.eps}
\caption{Variation of the potential with the charge of the black hole for
$\ell = 1$.$\,q=0.5,\, \mu=0.4,$ and $\omega=0.39$. The notable feature is that for small
enough background charge a well is present; but it ceases to exist when $Q$ is increased.}
\label{fig:pot_hbchar}
\end{figure}


%3) Variation of the field mass - qualitatively this is the same as for the Schwarzschild
%background. 


Figure~\ref{fig:pot_mass} shows the variation of $V_{eff}$ with the mass of the field for
fixed $Q,q$ and $\omega$. 
%The charge of the field is $q=0.5$ 
%and the frequency is $\omega = 0.1$.
For a massless scalar there is no well because the potential tends to zero
asymptotically. As the mass increases a potential well appears but above a
threshold, it vanishes. This behaviour of the potential is qualitatively
similar to that of the potential for an uncharged massive scalar field on a Schwarzschild
background \cite{Burt:2011pv}.

\begin{figure}[!ht]
\includegraphics[width=0.47\textwidth]{ms_Q0.8_q0.5_w0.1.eps}
\caption{Variation of the effective
potential with mass of the field for $\ell=1$, $q=0.5$, $Q=0.8$ $\omega=0.1$.
The asymptotic value at infinity is $\mu^2$. A potential well appears above a certain
minimum mass but it disappears above a certain maximum mass.}
\label{fig:pot_mass}
\end{figure}

Finally, the variation of the potential with $\omega$ is shown in Fig. \ref{fig:pot_w}. 
The trend is that both the height of the centrifugal barrier and the constant value near
the outer horizon increase with increasing frequency.


Let us close this section with two remarks concerning this effective potential, both of 
which originate from the fact that it depends on $\omega$. Firstly, since the frequency is
unknown, and only a discrete set of complex frequencies will be solutions of the wave
equation, the plots of the effective potential  should be taken just as guide to
understand the physical problem. Secondly, the fact that $V_{eff}$ depends on $\omega$
makes it unorthodox as compared to a standard proper potential in a Schr\"odinger-like
equation. In particular, the intuition that a potential well is necessary for the
existence of quasi-bound states (or bound states) is questionable, and indeed the results
we exhibit show that such states may exist even in the absence of a potential well for
this type of effective potential.


\begin{figure}[!ht]
\includegraphics[width=0.47\textwidth]{m0.4_Q0.8_q0.5_ws.eps}
\caption{Variation of the effective potential with the frequency for $M=1,\,q=0.5,\,
\mu=0.4 ,\, Q=0.8$, $\ell=1$.}
\label{fig:pot_w}
\end{figure}

%%%%%%%%%%%%%%%%%%%%%%%%%%%%%%%%%%%%%%%%%%%%%%%%%%
\section{Mirror quasi-bound states.}
\label{sec:Mbound-states}
%%%%%%%%%%%%%%%%%%%%%%%%%%%%%%%%%%%%%%%%%%%%%%%%%%
The states in the case of a black hole surrounded by a mirror are determined by the condition 
$\phi(r_{m}=0)$ where $r_{m}$ is the mirror radius. The mirrored sates for a rotating black hole
were considered in detail in Ref~\cite{Cardoso:2004nk}.
In order to obtain the states for a charged scalar field numerically, we proceed in the following
way:
We start
integrating the radial equation outward from
$r=r_{+}+\varepsilon$ ($\varepsilon>0$) with an arbitrary value $\omega$ and stop the integration at
the radius of the
mirror. 
This procedure gives us a value for the wave function at $r_m$, as function
of the frequency. Then, we look for the first root of this function (the ground state). When a
zero is reached, the scalar field vanishes. The integration is repeated varying the frequency until
the zero is reached with the desired precision and the final frequency is the frequency of the
mirrored state. 

Given a set of values of $q$ and $Q$ the observed behaviour of the real part of the frequencies is
that approaches the numerical value of mass of the field  and the imaginary part decreases
monotonically as $r_m$ increases.

For a given $Q$, if $r_m$ increases the real part of frequencies decreases, this is
a similar behaviour than in the non charged case, when the rotation parameter is fixed.

On the other hand for a given $q$, the real part of the frequency increases with $Q$ this is also
consistent with the picture
of the rotating case when $m$ is fixed, however the growing rate in the later the rate is very
small. For a charged BH the effect might be larger. 

\begin{figure*}
  \centering
  \begin{subfigure}[b]{0.5\textwidth}
    \centering
    \includegraphics[width=\textwidth]{{I-Q-all-ratio-1.0}.eps}
    % \caption{$\frac{q}{\mu}=1.0$}
    \label{fig:ratio1.0}
  \end{subfigure}%
  ~
  \begin{subfigure}[b]{0.5\textwidth}
    \centering
    \includegraphics[width=\textwidth]{{I-Q-all-ratio-1.2}.eps}
    % \caption{$\frac{q}{\mu}=1.2$}
    \label{fig:ratio1.2}
  \end{subfigure}
  \\
  \begin{subfigure}[b]{0.5\textwidth}
    \centering
    \includegraphics[width=\textwidth]{{I-Q-all-ratio-1.4}.eps}
    % \caption{$\frac{q}{\mu}=1.4$}
    \label{fig:ratio1.4}
  \end{subfigure}%
  ~
  \begin{subfigure}[b]{0.5\textwidth}
    \centering
    \includegraphics[width=\textwidth]{{I-Q-all-ratio-2.0}.eps}
    % \caption{$\frac{q}{\mu}=2.0$}
    \label{fig:ratio2.0}
  \end{subfigure}
  \caption{The imaginary part of the frequency plotted versus the radius of the mirror for various ratios of the scalar charge, $q$. The scalar mass has been fixed at$\mu=0.3$.}\label{fig:freq-plots-nolog}
\end{figure*}

In figures \ref{fig:freq-plots-nolog} we show the imaginary part of the frequency as a function of the
mirror radius for different values of the ratio $q/\mu$.
We see that when the ratio is unity, no amount of black hole charge will give positive values
for the imaginary part.
However, as the ratio increases, a range of mirror radii will have positive imaginary parts for a given $Q$.
We therefore focus on values for the scalar charge and mass where their ratio is larger than unity.


\begin{figure*}
  \centering
  \begin{subfigure}[b]{0.347\textwidth}
    \centering
    \includegraphics[width=\textwidth]{{I-mu-0.1-Q-all-q-0.6}.eps}
    % \caption{$\mu=0.1$}
    \label{fig:I-mu0.1-q0.6}
  \end{subfigure}%
  ~
  \begin{subfigure}[b]{0.33\textwidth}
    \centering
    \includegraphics[width=\textwidth]{{I-mu-0.2-Q-all-q-0.6}.eps}
    % \caption{$\mu=0.2$}
    \label{fig:I-mu0.2-q0.6}
  \end{subfigure}%
  ~
  \begin{subfigure}[b]{0.33\textwidth}
    \centering
    \includegraphics[width=\textwidth]{{I-mu-0.3-Q-all-q-0.6}.eps}
    % \caption{$\mu=0.3$}
    \label{fig:I-mu0.3-q0.6}
  \end{subfigure}%
  \\
  \begin{subfigure}[b]{0.347\textwidth}
    \centering
    \includegraphics[width=\textwidth]{{R-mu-0.1-Q-all-q-0.6}.eps}
    \caption{$\mu=0.1$}
    \label{fig:R-mu0.1-q0.6}
  \end{subfigure}%
  ~
  \begin{subfigure}[b]{0.33\textwidth}
    \centering
    \includegraphics[width=\textwidth]{{R-mu-0.2-Q-all-q-0.6}.eps}
    \caption{$\mu=0.2$}
    \label{fig:R-mu0.2-q0.6}
  \end{subfigure}%
  ~
  \begin{subfigure}[b]{0.33\textwidth}
    \centering
    \includegraphics[width=\textwidth]{{R-mu-0.3-Q-all-q-0.6}.eps}
    \caption{$\mu=0.3$}
    \label{fig:R-mu0.3-q0.6}
  \end{subfigure}%

  \caption{The imaginary and real part of $\omega$ drawn as a function of the mirror radius $r_m$ for various values of the black hole charge, $Q$, and the scalar mass, $\mu$. The charge of the scalar field is $q=0.6$.}\label{fig:freq-plots}
\end{figure*}

In figure~\ref{fig:freq-plots} we show the plots of both the imaginary and real
part of the frequency as a function of the mirror radius.
The three columns correspond to different values of the scalar mass
when the scalar charge has been fixed.
We see that the black hole charge which gives the maximum imaginary part changes when the scalar mass varies.

The key ingredient to have qausi-bound states with positive
imaginary parts is the mirror, the mass of the field acts only as a lower bound for the
real part of the frequencies. However, as the scalar mass increases, the magnitude
of the imaginary part of the frequency decreases, we have found that the black hole charge
that gives the
highest magnitude also changes from $Q=0.95$ to $Q=0.99$ when $\mu$ goes from $0.1$ to $0.3$.
However, increasing $\mu$ lowers the over all magnitude of the imaginary part.

\begin{figure*}
  \centering
  \begin{subfigure}[b]{0.33\textwidth}
    \centering
    \includegraphics[width=\textwidth]{{I-mu-0.1-Q-all-q-0.6-2}.eps}
    % \caption{$\mu=0.1$}
    \label{fig:I-mu0.1-q0.6-2}
  \end{subfigure}%
  ~
  \begin{subfigure}[b]{0.33\textwidth}
    \centering
    \includegraphics[width=\textwidth]{{I-mu-0.2-Q-all-q-0.8}.eps}
    % \caption{$\mu=0.2$}
    \label{fig:I-mu0.2-q0.8}
  \end{subfigure}%
  ~
  \begin{subfigure}[b]{0.33\textwidth}
    \centering
    \includegraphics[width=\textwidth]{{I-mu-0.3-Q-all-q-0.9}.eps}
    % \caption{$\mu=0.3$}
    \label{fig:I-mu0.3-q0.9}
  \end{subfigure}%
  \\
  \begin{subfigure}[b]{0.33\textwidth}
    \centering
    \includegraphics[width=\textwidth]{{R-mu-0.1-Q-all-q-0.6-2}.eps}
    \caption{$\mu=0.1$, $q=0.6$}
    \label{fig:R-mu0.1-q0.6-2}
  \end{subfigure}%
  ~
  \begin{subfigure}[b]{0.33\textwidth}
    \centering
    \includegraphics[width=\textwidth]{{R-mu-0.2-Q-all-q-0.8}.eps}
    \caption{$\mu=0.2$, $q=0.8$}
    \label{fig:R-mu0.2-q0.8}
  \end{subfigure}%
  ~
  \begin{subfigure}[b]{0.33\textwidth}
    \centering
    \includegraphics[width=\textwidth]{{R-mu-0.3-Q-all-q-0.9}.eps}
    \caption{$\mu=0.3$, $q=0.9$}
    \label{fig:R-mu0.3-q0.9}
  \end{subfigure}%

  \caption{The imaginary and real part of $\omega$ drawn as a function of the mirror radius $r_m$ for various values of the black hole charge, $Q$, the scalar charge, $q$, and the scalar mass, $\mu$.}\label{fig:freq-plots2}
\end{figure*}

In figures~\ref{fig:freq-plots2} we show the imaginary and real
part of the frequency as a function of the mirror radius when  both the
scalar mass and the scalar charge change. From these plots we notice that the magnitude of the
imaginary part increases as the scalar charge is increased. 
The values of $Q$ at which the maximum occurs are the same of figure~\ref{fig:freq-plots}.
As one can see, increasing the scalar charge increases the magnitude of
the real part but does not change the shape of the curve.
Furthermore, the  black hole charge has a slight effect on the curve in both
figures~\ref{fig:freq-plots} and~\ref{fig:freq-plots2}.

\begin{figure*}
  \centering
  \begin{subfigure}[b]{0.347\textwidth}
    \centering
    \includegraphics[width=\textwidth]{{I-mu-0.1-Q-all-q-0.9-Q}.eps}
    \caption{$q=0.9$}
    \label{fig:I-Q1.0-q0.9}
  \end{subfigure}%
  ~
  \begin{subfigure}[b]{0.33\textwidth}
    \centering
    \includegraphics[width=\textwidth]{{I-mu-0.1-Q-all-q-1.5-Q}.eps}
    \caption{$q=1.5$}
    \label{fig:I-Q1.0-q1.5}
  \end{subfigure}%
  ~
  \begin{subfigure}[b]{0.33\textwidth}
    \centering
    \includegraphics[width=\textwidth]{{I-mu-0.1-Q-all-q-2.0-Q}.eps}
    \caption{$q=2.0$}
    \label{fig:I-Q1.0-q2.0}
  \end{subfigure}%
  \caption{The imaginary and real part of $\omega$ drawn as a function of the mirror radius $r_m$ for various values of the black hole charge, $Q$, and the scalar charge, $q$. The scalar mass is fixed at $\mu=0.1$.}\label{fig:freq-plots-Q}
\end{figure*}

From figures \ref{fig:freq-plots-Q} we see that if the scalar charge is increased,
the black hole charge which gives the maximum imaginary part of the frequency
increases and eventually becomes the extremal case, $Q=M$.
This is similar to the behaviour mentioned above where changing the scalar mass, $\mu$, has the same effect.

From all of these results, we can then say that in order to get the maximum amplification of the scalar field,
the black hole should be extremal (or as close as one can get) while the scalar field should be as light as possible
but with the highest possible charge.
Once these three parameters have been fixed, it would simply be a matter of finding which position of the mirror 
gives the highest value.

%%%%%%%%%%%%%%%%%%%%%%%%%%%%%%%%%%%%%%%%%%%%%%%
\noindent{\bf{\em Acknowledgements.}}
%\end{acknowledgments}
%%%%%%%%%%%%%%%%%%%%%%%%%%%%%%%%%%%%%%%%%%%%%%%%%%%%%%%%%%%%%%%%%%%%%%%%%%%%
  This work was supported by the {\it NRHEP--295189} FP7-PEOPLE-2011-IRSES Grant, and by
  FCT -- Portugal through the project PTDC/FIS/116625/2010.


%%%%%%%%%%%%%%%%%%%%%%
%%%   REFERENCES   %%%
%%%%%%%%%%%%%%%%%%%%%%

\bibliographystyle{h-physrev4}
\bibliography{num-rel}


%%%%%%%%%%%%%%%
%%%   END   %%%
%%%%%%%%%%%%%%%
\end{document}
    
