\documentclass[a4paper]{article}
\usepackage{graphicx}
\usepackage{caption}
\usepackage{subcaption}
\begin{document}

\begin{figure}
  \centering
  \begin{subfigure}[b]{0.5\textwidth}
    \centering
    \includegraphics[width=\textwidth]{{I-Q-all-ratio-1.0}.png}
    % \caption{$\frac{q}{\mu}=1.0$}
    \label{fig:ratio1.0}
  \end{subfigure}%
  ~
  \begin{subfigure}[b]{0.5\textwidth}
    \centering
    \includegraphics[width=\textwidth]{{I-Q-all-ratio-1.2}.png}
    % \caption{$\frac{q}{\mu}=1.2$}
    \label{fig:ratio1.2}
  \end{subfigure}
  \\
  \begin{subfigure}[b]{0.5\textwidth}
    \centering
    \includegraphics[width=\textwidth]{{I-Q-all-ratio-1.4}.png}
    % \caption{$\frac{q}{\mu}=1.4$}
    \label{fig:ratio1.4}
  \end{subfigure}%
  ~
  \begin{subfigure}[b]{0.5\textwidth}
    \centering
    \includegraphics[width=\textwidth]{{I-Q-all-ratio-2.0}.png}
    % \caption{$\frac{q}{\mu}=2.0$}
    \label{fig:ratio2.0}
  \end{subfigure}
  \caption{The imaginary part of the frequency plotted versus the radius of the mirror for various ratios of the scalar charge, $q$, and the scalar mass, $\mu$, in both normal and log scale. $Q=0.8$ is cyan, $Q=0.0$ is black, $Q=0.95$ is green, $Q=0.99$ is red and $Q=0.997$ is blue. The ratios $0.4$, $0.8$ and $1.0$ give negative imaginary parts for all $Q$ so they are not shown here.}\label{fig:ratios2}
\end{figure}



\end{document}


